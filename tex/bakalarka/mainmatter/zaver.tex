%!TEX root = ../main.tex
\chapter{Záver}
\section{Ďalší možný vývoj aplikácie}
% \subsection{Dynamický herný svet}
\subsection{Pridanie bojového systému}
Jedno z možných vylepšení tohto frameworku, by mohla byť implementácia bojového systému a pridanie nepriateľov. Nepriateľ by bol pohyblivý, menší región, ktorý by automaticky pri strete s hráčom spustil súboj. Súboj by mohol byť teda aj ďalšou interakciou medzi hráčmi. Súboj by prebiehal formou malej hry na strane klienta, podľa nastavení serveru, ktorý by určoval ako hráčove atribúty ovplyvňujú boj. 

\subsection{Hráčmi vytváraný a hodnotený obsah}
Zaujímavým rozšírením tejto práce by mohla byť možnosť hráčov pomocou herného klienta pridávať obsah do hry, ak nie je inak nastavené na hernom serveri. Zároveň by mali používatelia možnosť hodnotiť tento obsah podľa kvality, a tým administrátorov informovať o kvalitných častiach hry, ktoré je vhodné ponechať.

\subsection{Komunikácia pomocou NFC}
Hráči môžu medzi sebou komunikovať, a teda si odovzdávať objekty pomocou technológie Bluetooth. Ďalším vhodným spôsobom komunikácie by mohlo byť pomocou NFC technológie, ktorá zatiaľ ešte nie je až tak rozšírená, no urýchlila by tento proces. Pri stretnutí dvoch hráčov by si obaja zapli NFC na svojich zariadeniach a priložili ich k sebe. Oproti existujúcej implementácii pomocou Bluetoothu nie je potreba párovať zariadenia, či ich vyberať zo zoznamu. 

\section{Iné možnosti využitia}
\subsection{Tvorba šifrovacích hier}
Framework môže byť využitý na tvorbu šifrovacích hier. Stačí vytvoriť žiadané úlohy a vytlačiť jednotlivé QR kódy na stanoviská, ktoré bývajú v týchto hrách. Iným možným postupom je jednotlivé stanoviská označiť regiónmi a napojiť k nim úlohy, ktoré sa automaticky spustia po vstupe na územie regiónu.


\subsection{Prideľovanie úloh vzhľadom na polohu}
Ďalším možným využitím je prideľovanie úloh používateľom, ktoré budú prideľovať administrátori. Tiež je možné zadanie úloh pomocou regiónov, ktoré spúšťajú automaticky úlohy pre používateľov, čím je zaistené pridelenie úlohy najvhodnejšiemu členovi tímu vzhľadom na jeho polohu. Administrátor tiež môže zadať úlohu pre konkrétneho používateľa manuálne. Používateľov klient, ktorému je úloha zadaná sa automaticky aktualizuje a informovaný používateľ môže úlohu vykonať.


\subsection{Tvorba turisticko-historickej prehliadky}
Ako príklad si môžeme uviesť: Turista si stiahne mobilnú aplikáciu pre Android telefón, zapne GPS a pripojí sa na server. Po získaní informácii zo servera sa dozvie, že tam kde práve teraz stojí bol pred mnohými rokmi chrám. Prečíta si o jeho histórii a môže si pozrieť ako vyzeral. Taktiež sa mu automaticky spustí úloha - navštívenie ďalšieho bodu prehliadky. Tvorcovia prehliadky mohli zanechať QR kódy s odmenami popri neďalekej soche. Keď turista načíta  QR kód, dostane do inventáru sošku. Potom sa o nej môže dozvedieť bližšie informácie, aj keď už okolie sochy opustí. 

\section{Zhrnutie}
Cieľom tejto bakalárskej práce bolo vytvoriť nástroj na tvorbu GPS online hier. Tento cieľ sa podarilo splniť, keďže vznikol nástroj, ktorý umožňuje používateľovi jednoduché a komplexné riešenie bez potreby znalostí programovacích jazykov. Toto riešenie ponúka herného klienta, aplikáciu herného servera spolu s nástrojmi na tvorbu herného sveta. \

Aplikácia herného klienta bola vyvíjaná na Android telefóne Samsung S4 Mini I9195 s verziou Android 4.2. Webová aplikácia bola vyvinutá na serveri s verziou PHP 5.3 a MySQL 5.1. \

Pre ďalší vývoj aplikácie by bolo vhodné implementovať bojový systém a hráčom dať možnosť pridávať obsah do hry a možnosť hodnotiť takto pridaný obsah priamo z klienta.