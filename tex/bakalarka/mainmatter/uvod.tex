%!TEX root = ../main.tex
\chapter{Úvod}
\paragraph{}
Hry sú súčasťou ľudstva už od odvekých dôb. Mnohé, od kockových a kartových cez doskové až po kolektívne športové hry vonku, prinášajú ľudom radosť a sú určitou formou zábavy. Často však pomáhajú rozvíjaniu osobnosti, vzniku nových priateľstiev, či pri vzdelávaní. S príchodom moderných technológií, ako sú počítače a internet, sa vytvorilo nové herné odvetvie počítačových hier, ktoré často ponúkajú možnosť hrať hru spolu s ostatnými hráčmi. Počítačové hry predstavujú hráčom nové virtuálne svety. V týchto svetoch môžu byť na neobyčajných miestach, súčasťou neuveriteľných príbehov a spolu riešiť úlohy, či porovnávať svoje zručnosti medzi sebou. Hráči, často dokážu presedieť desiatky hodín týždenne za počítačom, tabletom či smartfónom, a preto je možné čoraz častejšie vidieť kedysi plné ihriská, zívať prázdnotou. 

\paragraph{}
Cieľom tejto bakalárskej práce je vytvoriť nástroj, pomocou ktorého budú vznikať hry, ktoré budu môcť byť lákavou zmesou zaujímavého deja, pohybu, hľadania tajných indícií a rozmýšľania nad rôznymi úlohami. Tento nástroj má odbremeniť tvorcov takýchto hier, od potreby znalostí programovania na rôznych platformách a riešenia rôznych implementačných problémov. Pri tvorbe vlastných hier sa preto môžu sústrediť na podstatné stránky a veľmi rýchlo vytvoriť vlastnú hru. Tento nástroj by mal umožňovať spoluprácu viacerých autorov pri tvorbe hry. Herný klient by mal byť univerzálny pre všetky hry vytvorené pomocou tohto nástroja. Mal by byť schopný posielať informácie o polohe zariadenia na herný server, ktorý je súčasťou nástroja. Ako odpoveď zo serveru by mal dostať informácie o aktuálnom hernom svete a vedieť ich zobraziť prívetivou formou pre hráča. Systém by mal umožňovať vytváranie drobných skrýš, ktoré by mohli obsahovať herné objekty alebo úlohy. Pri tvorbe hier by mal byť kladený veľký dôraz na atmosféru, dej a prepracovanosť úloh, s nimi spojenými, aby výsledný zážitok z hry bol čo najpozitívnejší. 

\paragraph{}
Táto práca je rozdelená do viacerých častí, v ktorých sa snaží vysvetliť základné pojmy, priblížiť prostredie hier a ich tvorby. Ďalšie kapitoly sa zaoberajú analýzou problémov a hľadaní riešení, spolu s popisom konkrétnej implementácie.