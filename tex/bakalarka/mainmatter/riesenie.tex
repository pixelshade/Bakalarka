%!TEX root = ../main.tex
\chapter{Návrh riešenia}


\section{Použité technológie}


\section{Štruktúra aplikácie}


\section{Členenie hry}
\subsection{Administrátor}
Je používateľ webovej aplikácie, ktorý ma prístup do administrátorskej sekcie na servery. Tam môže vytvárať, upravovať a mazať jednotlivé vlastnosti herného sveta. Tieto vlastnosti môžu byť regióny, úlohy, jednotky, objekty vo svete. 

\subsection{Klient - hráč}
Je používateľ, ktorý používa aplikáciu na mobilnom zariadení. Hráč hrá za virtuálnu postavu v hernom svete. Pri pohybe v realnom svete sa zistuje hráčova aktuálna poloha pomocou GPS a je zaslaný dopyt na server s aktuálnou polohou. Zo servera dostane vlastnosti herného sveta pre aktuálnu polohu.

\subsection{Herny svet}
\paragraph{}
Herny svet je tvoreny regionmy. Sú to plochy v priestore, v ktorých sa može nachádzať hráč. Hráč pohybom v reálnom svete sa pohybuje zároveň aj v tom hernom a na mape môže vidieť v akom hernom regione sa nachádza. Regióny sú často spojené s úlohami, ktoré možno vykonať za odmenu. Úlohy môžu byť také, v ktorých hráč musí poraziť určitý počet nepriateľov, získať a nájsť určité predmety, odpovedať na určitú otázku a tieto úlohy môžu byť ohraničené na čas, za ktorý musia byť splnené ináč budú neúspešné. V hernom svete sa na roznych poziciách možu nachádzat a pohybovať jednotky z hry. Tie keď sa dostanú do kontaktu s hráčom môžu vyvolať súboj. V hernom svete tiež sú umiestnené QR kody a na niektorých miestach i NFC tagy, ktoré pridávajú do hry detailnejší pohľad na svet. Možu predstavovať herné objekty ako zbrane, pasce, vybavenie ale i informácie o prostredí, príbehu či úlohy. 

\paragraph{}
Postava má určité vlastnosti. Jednou z hlavných sú životy, pri ktorých počet klesnúci na nulu znamená porážku v súboji s nepriateĺom. Postava má peniaze, ktoré môže získať plnením úloh či porážaním nepriatelov. Môže si za ne kúpiť zbrane či iné vybavenie, ktoré mu môže vylepšovať atribúty. Pomocou tychto atribútov sa v súboji zisťuje ako prebieha súboj. Ďalej má skúsenosti a schopnosti. Schopnosti, ktoré sa odomykajú na používanie hráčovi s pribúdajúcimi skúsenosťami. Tieto schopnosti môže použiť v súboji, k zlepšeniu svojich šancí na porazenie nepriateľov. 

\section{Herný príklad}
Hráč si zapne herného klienta na mobilnom android zariadení. Prečíta si informáciu, o tom že sa nachádza v bažinách, o ktorých sa traduje, že sa tam nachádzajú trolovia. Môže si popozerať obrázky bažín ktoré lepšie navodia atmosféru. Dozvie sa aj o úlohe, ktorú môže splniť. Poraziť trola, ktorý nivočí okolie. Najprv ho musí nájsť. Nájde QR kód, ktorý keď načíta mu povie bližšie informácie ako ho poraziť a kde ho hľadať. Musí preto nájsť čarovný meč ktorý sa nachádza obďaleč. Po nájdení tohto chýbajúceho članku k jeho víťazstvu spĺňa úlohu a získava odmenu.
\paragraph{}
Administrátor cez webové rozhranie na servery vytvorí región bažín na určitej ploche. Pridá do nej úlohu o zničení trola a o následnej odmene ak splní hráč podmienku a priniesie čarovný meč . Pridá ešte pomocný QR kód pre lahšie nájdenie meča a samotný meč.\

\paragraph{}
Z hernej ukážky môžeme povedať, že vysledné hry budu môcť čerpať časť čŕt z larpov, kde sa hráči vžíjú do svojich postáv a prechádzajú určitým príbehom. Tiež geocachingu, kde hráči hľadajú kešky(správy či iné malé prekvapenia), ktoré pre nich zanechali ostatný na určitej GPS pozícii.

\section{Minimálne požiadavky}
\subsection{Klient}
\subsection{Server}


\section{Spôsoby riešenia problémov}
\subsection{Tvorba herného sveta}
\subsection{Univerzálnosť klienta}
\subsection{Vytvorenie redakčného systému}
\subsection{Komunikacia klient-server}
\subsection{Spušťanie pomocou QR kódov}
\subsection{Komunikácia medzi hráčmi}



\section{Ciele}
Cielom tejto práce je vytvoriť prístupný a jednoduchý nástroj na tvorbu multiplayerových online hier, ktoré. Taktiež prenechať priestor pre možnosť vytvoriť aplikáciu, ktorá bude môcť  ktoré všetkých tých hráčov, ktorí presedeli desiatky hodín za počítačom vyťiahnuť von a vydať
